\documentclass{beamer}

\usepackage{beamerthemesplit}

\title{LightCTL}
\date{\today}
\author{Spencer Krum}
\begin{document}

\frame{\titlepage}

%Slide
\section{Introduction}
\subsection{Prerequesites}
\frame
{
	\frametitle{Prerequisites}
	\begin{itemize}
		\item Open Source Software Class
		\item Basic Circuits
		\item Basic C
		\item Basic Python
	\end{itemize}
}

%Slide
\subsection{Overview of Topics}
\frame
{
	\frametitle{Overview of Topics}
	\begin{enumerate}
		\item Dr. Wamser's Lab
		\item The Circuit
		\item The Two Programs
		\begin{itemize}
			\item Dumb Teensy
			\item Smart Teensy
		\end{itemize}
		\item Microcontrollers and You
	\end{enumerate}
}

%Slide
\subsection{Wamser Lab}
\frame
{
	\frametitle{Wamser Lab}
	\begin{itemize}
		\item Dr. Wamser's group - SB1 324
		\item Organic Solar Cells - Artifical Photosynthesis
		\item Cells come out of the oven 2mm thick and 2.5cm square
		\item Cells are categorized by IV curves under constant illumination
		\item Lamp used was powered by batteries - poor consistency
	\end{itemize}
}

%Slide
\section{Hardware}
\subsection{Microcontroller}
\frame
{
	\frametitle{Teensy++}
	\begin{itemize}
		\item Teensy++ is an Arduino-like microcontroller
		\item Atmel AT90USB1286 - 8bit - 16Mhz - mini USB
		\item Cost - \$18 or \$24 
		\item I have experience with it because of the UROV project
		\item Two types of output, PWM and digital I/O
		\item www.pjrc.com has Teenys and example code/tutorials
		\item Example code includes serial over USB
		\item Code written in C, gcc-avr package available as F/OSS
	\end{itemize}
}

%Slide
\subsection{Power}
\frame
{
	\frametitle{Power}
	\begin{itemize}
		\item Lamp runs on 12V DC
		\item Formerly powered off batteries
		\item Now AC/DC transformer feeds into 12V synchronous regulator board
		\item Teensy outputs 8bit PWM signal to $V_{adj}$ pin on regulator board
		\item Result is smooth control between 8 and 12.5 V DC
	\end{itemize}
}

%Slide
\subsection{Input}
\frame
{
	\frametitle{Input}
	\begin{itemize}
		\item CdS (Cadmium Sulfide) photoresistor 
		\item LDR is one half of a voltage divider spanning 5V
		\item Analog voltage is read at the middle of the divider
		\item Teensy has internal 10bit ADC
		\item External analog reference not used - option for future improvement
	\end{itemize}
}
%Slide
\subsection{Output}
\frame
{
	\frametitle{Circuit}
	\begin{itemize}
		\item PWM is pulse width modulated
		\item PWM signal is hooked to $V_{adj}$ pin on regulator board
		\item 5 digital I/O pins control 7 seg displays
		\item 1 I/O pin uses NAND IC(thanks Phil!) to multiplex across 2 displays
		\item 4 I/O pins use BCD to a BCD-7Segment display decoder to run 7-Seg
		\item Future version will use analog dial to display output
	\end{itemize}
}
%Slide
\section{Software}
\subsection{Two Versions}
\frame
{
	\frametitle{Two Versions}
	\begin{itemize}
		\item In short: there are two versions of LightCTL, ideally they do the same thing
		\item One, Smart Teensy, runs a closed loop control system on the teensy
		\item The other, Dumb Teensy, runs a closed loop control system on the controlling computer in python
	\end{itemize}
}
%Slide
\subsection{Teensy Programming Basics}
\frame
{
	\frametitle{Teensy Programming Basics}
	\begin{itemize}
		\item Teensy is a device with a little over 8K memory
		\item instead of int, use uint8\_t
		\item Registers control everything
		\item Bit twiddling 
		\item DDRB $|= (1<<4)|(1<<3)$;
	\end{itemize}
}
%Slide
\subsection{Control Loop}
\frame
{
	\frametitle{Control Loop}
	\begin{itemize}
		\item Simple control loop uses PID
		\item P is for Proportion
		\item I is for Integral
		\item D is for Derivative
	\end{itemize}
}
%Slide
\subsection{Serial}
\frame
{
	\frametitle{Serial}
	\begin{itemize}
		\item PJRC supplied code has Teensy appear as serial device
		\item On Linux this looks like /dev/ttyACM0
		\item For the UROV greghaynes++ wrote a simple syntax for serial communication
		\item Python:
		\begin{itemize}
			\item import serial
			\item x = serial.Serial('/dev/ttyACM0')
			\item x.write('\textbackslash x04\textbackslash x03\textbackslash xFF\textbackslash n')
		\end{itemize}
		\item First char indicates job, 4 is PWM, 7 is Sensor
		\item Second char indicates which pwm/sensor
		\item Third char (pwm only) indicates value
		\item Newline delimited
	\end{itemize}
}
%Slide
\section{Outro}
\subsection{Next}
\frame
{
	\frametitle{Whats next for LightCTL}
	\begin{itemize}
		\item Combine Dumb Teensy and Smart Teensy
		\item Make it work with LabView
		\item Web interface/control
		\item Printed Circuit Board
	\end{itemize}
}


%Slide
\subsection{Microcontrollers and You}
\frame
{
	\frametitle{You really can hack on a Microcontroller}
	\begin{itemize}
		\item YOU! You really can.
		\item There is a ton of example code for all kinds off microcontrollers
		\item There are a ton of example circuts for everything imaginable
		\item There are people in \#cschat, irc.cat.pdx.edu, that can help you(I'm but one)
		\item Often getting something simple to work can be empowering
		\item Bart teaches a class
		\item and finally... \\.
	\end{itemize}
}


%Slide
\subsection{Gotcha}
\frame
{
	\frametitle{Circuit}
	\begin{itemize}
		\item If the Physics Student Can ... you better be able to!
	\end{itemize}
}
%Slide
\subsection{Thanks}
\frame
{
	\frametitle{Thanks!}
	\begin{itemize}
		\item Thanks for listening to me ramble about LightCTL
		\item You can get it at www.github.com/spencer-krum/LightCTL
		\item Thanks to Bart Massey for OSS class
		\item Thanks to Greg Haynes for writing a code base to work off
		\item Thanks to Erik Sanchez for the circuits stuff
		\item Thanks to Phil for the NAND/help with circuits
		\item Thanks to Devin Quirozoliver for lots of things, but mostly his vimrc and the beamer I hacked this one out of
	\end{itemize}
}

\end{document}
